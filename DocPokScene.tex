\documentclass[10pt,a4paper]{article}
\usepackage[utf8]{inputenc}
\usepackage{amsmath}
\usepackage{amsfonts}
\usepackage{amssymb}
\usepackage{graphicx}
\usepackage{listings}
\usepackage[margin=0.5in]{geometry}

\author{Oleg Loshkin}
\title{PokScene JSON Format}
\lstset{
    breaklines=true,
    tabsize=2
}

\begin{document}

\maketitle
\noindent \textbf{Doc git repo: https://github.com/LoshkinOleg/DocPokScene}

\section{Files location}
Currently .pokscene files are located under \textbf{"save/Scenes"} instead of AppData. This will need to be addressed at some point since scenes need to be in the Switch as assets.

\section{Arrays}
\textbf{Arrays are fields that look like "someField": []}. These are preceeded by the \textbf{length of the array} with a name of \textbf{fieldName + "Count" like so: "someFieldCount": 0}.\\\\
\textbf{WARNING: Array indices begin at 1 instead of 0.} Value of 0 is reserved for the default index that indicates it is not a valid one (ex: game object with "parent": 0 field signifies it has no parent. Game object with "parent: 1" signifies that the game object at the first position in the array is it's parent.).

\section{Empty or default fields}
\textbf{Empty or default fields} such as "myTexturePath": "" or "parent":0 \textbf{are automatically removed upon .pokconvertiblescene file generation by UPDC on Unity's side} as long as the string value describing what the default field looks like is written in "Empty Fields" tab in UPDC. \textbf{If you don't find a particular field somewhere, it means it is set to the default value}.\\\\
\textbf{WARNING: These are exceptions to this rule:}
\begin{itemize}
\item All values and sub-values of a game object's "transform".
\item A game object's "instanceId" (Unity's unique identifier for a game object instance).
\item A "model"'s "diffuseColor".
\item A "model"'s "roughness".
\item A chunk's "position".
\item A chunk's "size". 
\end{itemize}

\noindent The values listed above are always present as long as their parent values are present (every model will have a color for instance but not all game objects will have a model).

\section{Prefabs}
\textbf{Nested prefabs are not supported.}\\
TODO: Prefabs are not yet handeled by ConvertibleSceneToPokSceneTool on Poke's side!

\newpage

\section{Overall Format}
The jsons in this documentation are formatted for easy reading.
\begin{lstlisting}
{
	"resources": // Hash - filename pairs for ressources to load. Can be empty.
	{
		...	
	},
	
	"gameObjectsCount": ...,
	
	"gameObjects": // Game object and prefab instances.
	[
		...	
	],
	
	"skybox": // Textures used by the skybox in scene.
	{
		...	
	},
	
	"chunksCount": ...,
	
	"chunks": // Scene's chunks with indices to game objects and prefabs they contain.
	[
		...	
	],
	
	"chunkTimeEventsCount": ...,
	
	"chunkTimeEvents": // ChunkTimeEvents with indices to chunks.
	[
		...	
	],
	
	"chunkTriggerEventsCount": ...,
	
	"chunkTriggerEvents": // ChunkTriggerEvents with indices to chunks.
	[
		...	
	]
}
\end{lstlisting}

\newpage

\section{resources}
\begin{lstlisting}
"resources":
	{
		"texturesCount": 1,
		"textures": // Enum index corresponding to ResourceType defined in resource_type.h.
		[
			{
				"hash": 68431, // Generated with engine's hasher from the string below.
				"name": someTexture.png"
			}		
		],
		"meshesCount": 1,
		"meshes": 
		[
			{
				"hash": 68431,
				"name": someMesh.obj"		
			}		
		]
	}
\end{lstlisting}

\newpage

\section{gameObjects}
\textbf{The first game object shows a game object with only "parent" field removed} since it has no parent. \textbf{The second game object shows one with all fields set to default values except "parent"} which is set to the index of the first one. (NOTE: see "Arrays" section of the document as to why indices start at 1 instead of 0.)\\
\textbf{The third game object shows an instance of a prefab that is a child of game object} at index 1.
\begin{lstlisting}
"gameObjectsCount": 3,
"gameObjects":
[
	{ // This is a game object at index 1.
		"isActive": true, // Only present if "isActive" is true, else "isActive" is removed.
		"transform": // Values are relative to parent. ALWAYS PRESENT
		{
			"position":
			{
				"x": 50.0,
				"y": 50.0,
				"z": 50.0
			},	
			"rotation":
			{
				"x": 50.0,
				"y": 50.0,
				"z": 50.0
			},
			"scale":
			{
				"x": 50.0,
				"y": 50.0,
				"z": 50.0
			}
		},
		"model":
		{
			"meshName": 843148, // Hash of the resource located in the "resources" section.	
			"diffuseMapName": 32156,
			"normalMapName": 2148435,
			"occlusionMapName": 410216, // Default value of 0.
			"roughness": 1.0, // ALWAYS PRESENT
			"diffuseColor": // ALWAYS PRESENT
			{
				"r": 1.0,
				"g": 1.0,
				"b": 1.0,
				"a": 1.0
			}
		},
		"spline": 
		{
			"pointsCount": 2,
			"points":
			[
				{
					"x": 1.0,				
					"y": 25.0,
					"z": 30.0
				},
				{
					"x": 2.0,				
					"y": 25.0,
					"z": 30.0
				}
			]		
		},
		"collider":
		{
			"size":
			{
				"x": 1.0, // // ALWAYS PRESENT,  even if at 0.0.
				"y": 25.0,
				"z": 1.0
			},
			"radius": 35.0 // Default value of 0 .
		}
	},
	{ // Game object instance at index 2.
		"parent": 1	
	},
	{ // Prefab instance at index 3.
		"prefab": 5, // Id of the prefab that corresponds to the id in .pokprefab.
		"parent": 1, // This prefab instance is a child of GO at index 1.
		"isActive": true,
		"transform":
		{
			"position":
			{
				"x": 50.0,
				"y": 50.0,
				"z": 50.0
			},	
			"rotation":
			{
				"x": 50.0,
				"y": 50.0,
				"z": 50.0
			},
			"scale":
			{
				"x": 50.0,
				"y": 50.0,
				"z": 50.0
			}
		}
	}
]
\end{lstlisting}

\newpage

\section{skybox}
A skybox is composed of \textbf{6 textures for each face of the skybox}.
\begin{lstlisting}
"skybox":
{
	"left": "skyboxTexture0.png",
	"right": "skyboxTexture1.png",
	"bottom": "skyboxTexture2.png",
	"top": "skyboxTexture3.png",
	"back": "skyboxTexture4.png",
	"front": "skyboxTexture5.png"
}
\end{lstlisting}

\section{chunks}
\textbf{"gameObjects" and "prefabs" fields are always present, even if empty}. This is because chunks may be empty at the start of a scene until something moves within it's bounds during gameplay (like the player).
\begin{lstlisting}
"chunksCount": 2
"chunks":
[
	{ // Chunk at index 1.
		"position":
		{
			"x": 10.0, // Default value of 0.0 .
			"y": 25.0,
			"z": 30.0
		},
		"size":
		{
			"x": 2.0, // Default value of 1.0 .
			"y": 3.0,
			"z": 50.0
		},
		"gameObjects": // ALWAYS PRESENT. Has no default value.
		{
			3, // Index of the game object in "gameObjects" field. Default value of 0.
			25,
			1
		},
		"prefabs": // ALWAYS PRESENT. Has no default value.
		{
			1, // Index of the game object in "prefabs" field. Default value of 0.
			13
		}
	}
]
\end{lstlisting}

\newpage

\section{chunkTimeEvents}
\begin{lstlisting}
"chunkTimeEventsCount": 1
"chunkTimeEvents":
[
	{ // Event at index 1.
		"time": 34.126, // Default value of 0.0 .
		"toShowCount": 1, // Default value of 0.
		"toShow":
		[
			2 // Index of the chunk to show in "chunks" field.
		],
		"toActivateCount": 1,
		"toActivate":
		[
			2		
		],
		"toHideCount": 2,
		"toHide":
		[
			1,
			3		
		],
		"toDestroyCount": 1,
		"toDestroy":
		[
			1
		],
	}
]
\end{lstlisting}

\section{chunkTriggerEvents}
\begin{lstlisting}
"chunkTriggerEventsCount": 1
"chunkTriggerEvents":
[
	{ // Event at index 1.
		"chunk": 2, // Index of the chunk that triggers this event. Default value of 0.
		"toShowCount": 1, // Default value of 0.
		"toShow":
		[
			2 // Index of the chunk to show in "chunks" field.
		],
		"toActivateCount": 1,
		"toActivate":
		[
			2		
		],
		"toHideCount": 2,
		"toHide":
		[
			1,
			3		
		],
		"toDestroyCount": 1,
		"toDestroy":
		[
			1
		],
	}
]
\end{lstlisting}

\end{document}